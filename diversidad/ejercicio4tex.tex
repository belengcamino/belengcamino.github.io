\documentclass[11pt,a4paper]{article}
\usepackage[spanish]{babel}
\usepackage[utf8]{inputenc}
\usepackage[]{gensymb}
\usepackage{amsmath, amssymb, dsfont}
\usepackage{mathtools}
%\usepackage{ tipa }

\setlength{\parindent}{0pt}
\decimalpoint


\begin{document}
\centerline{{\Large \textbf{Examen Funciones Matemáticas I}}}
\centerline{{\Large \textbf{1ºB Bachillerato de Ciencias}}}

\begin{enumerate}
\item Dadas las funciones $f(x)=\sqrt{x-2}$, $g(x)=\dfrac{x-1}{x+3}$ y $h(x)=\log\left(\dfrac{1}{x}\right)$ calcula:
\begin{itemize}
     \item sus dominios
     \item la función inversa de $g(x)$
     \item la composición $f(x)\circ g(x)$
\end{itemize}

\item Calcula los siguientes límites (en caso de no existir, explica por qué):
\begin{itemize}
     \item $\displaystyle\lim_{x\rightarrow 5}\dfrac{3x-15}{\sqrt[]{x+4}-3}$
     \item $\displaystyle\lim_{x\rightarrow 1}\dfrac{x^3-3x^2+3x-1}{x^3-x^2-x+1}$
     \item $\displaystyle\lim_{x\rightarrow \infty}\dfrac{x^3-3x^2+3x-1}{x^3-x^2-x+1}$
     \item $\displaystyle\lim_{x\rightarrow +\infty}\left(\dfrac{x^2-1}{3x-1}-\dfrac{x^3-x^2+1}{3x^2}\right)$
\end{itemize}


\item Encuentra razonadamente la expresión analítica de una función racional que cumpla los siguientes puntos y además realiza una representación gráfica de la misma.
\begin{itemize}
    \item Tiene una discontinuidad evitable en $x=3$
    \item Tiene asíntotas verticales en $x=1$ y $x=-1$
    \item Tiene asíntota horizontal en $y=2$
\end{itemize}

\item Dada la función
\begin{equation*}f(x)=\left\{
\begin{array}{rl}
\dfrac{1}{x-1} & \text{si } x< 0 \\
2x-1 &\text{si }0\leq<x<2 \\
x^2-4x+3 & \text{si }x\geq 2
\end{array} 
\right.
\end{equation*}
Estudia su continuidad y explica el tipo de discontinuidad, en caso de que lo presente, en los puntos $x=0$ y $x=2$.


\item Representa gráficamente la función $f(x)=\dfrac{4x-3}{x-2x^2-4x+3}$.

\end{enumerate}


\newpage
\centerline{{\Large \textbf{Examen Funciones Matemáticas I}}}
\centerline{{\Large \textbf{1ºB Bachillerato de Ciencias}}}

\begin{enumerate}
\item Dadas las funciones $f(x)=\sqrt{x-2}$, $g(x)=\dfrac{x-1}{x+3}$ y $h(x)=\log\left(\dfrac{1}{x}\right)$ calcula:
\begin{itemize}
     \item sus dominios
     \item la función inversa de $g(x)$
     \item las composiciones $f(x)\circ g(x)$, $g(x)\circ f(x)$, $h(x)\circ f(x)$ y sus dominios.
\end{itemize}

\item Calcula los siguientes límites (en caso de no existir, explica por qué):
\begin{itemize}
     \item $\displaystyle\lim_{x\rightarrow 5}\dfrac{3x-15}{\sqrt[]{x+4}-3}$
     \item $\displaystyle\lim_{x\rightarrow 3}\left(\dfrac{1}{x-2}\right)^{\frac{1}{x-3}}$
     \item $\displaystyle\lim_{x\rightarrow +\infty}\left(\dfrac{x^2-1}{3x-1}-\dfrac{x^3-x^2+1}{3x^2}\right)$
     \item $\displaystyle\lim_{x\rightarrow 5}\displaystyle\sqrt[]{\dfrac{x^2-4x+6}{x-5}}$
\end{itemize}


\item Encuentra razonadamente la expresión analítica de una función racional que cumpla los siguientes puntos y además realiza una representación gráfica de la misma.
\begin{itemize}
    \item Tiene una discontinuidad evitable en $x=3$
    \item Tiene asíntotas verticales en $x=1$ y $x=-1$
    \item Tiene asíntota horizontal en $y=2$
\end{itemize}

\item Dada la función
\begin{equation*}f(x)=\left\{
\begin{array}{rl}
\dfrac{4x}{x+3} & \text{si } x\leq -1 \\
\dfrac{x^2+x-2}{x^2-1} &\text{si } -1<x<1 \\
mx-2 & \text{si }x\geq 1
\end{array} 
\right.
\end{equation*}
Estudia su continuidad y encuentra el valor de $m$ para que sea continua en $x=1$.


\item Representa gráficamente la función $f(x)=\left|\dfrac{x+3}{x-2}\right|$.

\item Encuentra tres intervalos que no tengan elementos en común (disjuntos), en cada uno de los cuales la ecuación $2x^4-14x^2+14x-1=0$ tenga una solución.


\end{enumerate}

\end{document}