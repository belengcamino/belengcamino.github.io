% !TEX encoding = UTF-8 Unicode
%-------------------------------------------------------------------------
%
%  Ejemplo: Bibliografía con LaTeX en el entorno thebiblliography 
%
%-------------------------------------------------------------------------
\documentclass[11pt,a4paper]{article}
\usepackage[utf8]{inputenc}
\usepackage[spanish]{babel}


%-------------------------------------------------------------------------
% Título
%
\title{Alimentos transgénicos
\thanks{Resumen del artículo publicado en la revista Science}}
\author{Nombre del Autor
\thanks{email@us.es}}
\date{}



%-------------------------------------------------------------------------
%-------------------------------------------------------------------------
\begin{document}
\maketitle




\begin{abstract}
Este trabajo trata sobre los alimentos genéticos. Se exponen los beneficios que se obtienen con su producción: incremento de la productividad, resistencia a plagas, nuevas características de calidad, \dots. 

Se exponen asimismo algunas de las controversias que suscitan: aparición de bacterias patógenas, el empleo de DN de especias distintas, aparición de reacciones alérgicas, etc.

Por último, se tratan las cuestiones de propiedad intelectual y patentes.
\end{abstract}




%----------------------------------------------------------------------
%\chapter{Alimentos transgénicos}

%-------------------------------------------------------------------------
\section{Introducción}

Los alimentos sometidos a ingeniería genética o alimentos transgénicos son aquellos que fueron producidos a partir de un organismo modificado genéticamente mediante ingeniería genética. Dicho de otra forma, es aquel alimento obtenido de un organismo al cual le han incorporado genes de otro para producir una característica deseada. En la actualidad tienen mayor presencia alimentos procedentes de plantas transgénicas como el maíz, la cebada o la soja.


\begin{table}[h!]
\centering
\begin{tabular}{|l|c|c|}
\hline
     &  Transgénicos (\%)  & No transgénicos (\%)  \\
\hline\hline
Maíz  &    58  &   42 \\
\hline
Cebada &   33   &  67   \\
\hline
Soja   &  66   & 44 \\
\hline
\multicolumn{3}{l}{\footnotesize Datos completamente inventados}
\end{tabular}
\caption{Datos sobre la producción transgénica en el año 1500}
\label{tabla.transg}
\end{table}

La ingeniería genética o tecnología del ADN recombinante es la ciencia que manipula secuencias de ADN ---que normalmente codifican genes--- de forma directa, posibilitando su extracción de un taxón biológico dado y su inclusión en otro, así como la modificación o eliminación de estos genes. En esto se diferencia de la mejora clásica, que es la ciencia que introduce fragmentos de ADN (conteniendo como en el caso anterior genes) de forma indirecta, mediante cruzamientos dirigidos.
 La primera estrategia, la de la ingeniería genética, se circunscribe en la disciplina denominada biotecnología vegetal. Cabe destacar que la inserción de grupos de genes mediante obtención de híbridos ---incluso de especies distintas--- y otros procesos pueden realizarse mediante técnicas de biotecnología vegetal que no son consideradas ingeniería genética, como puede ser la fusión de protoplastos.


La mejora de las especies que serán usadas como alimento ha sido un motivo común en la historia de la Humanidad. Entre el 12000 y 4000 a. de C. ya se realizaba una mejora por selección artificial de plantas. Tras el descubrimiento de la reproducción sexual en vegetales, se realizó el primer cruzamiento intergenérico (es decir, entre especies de géneros distintos) en 1876. En 1909 se efectuó la primera fusión de protoplastos, y en 1927 se obtuvieron mutantes de mayor productividad mediante irradiación con rayos X de semillas. Finalmente, en 1983 se produjo la primera planta transgénica y en 1994 se aprobó la comercialización del primer alimento modificado genéticamente.


En el año 2007, los cultivos de transgénicos se extienden en 114,3 millones de hectáreas de 23 países, de los cuales 12 son países en vías de desarrollo.
 En el año 2006 en Estados Unidos el 89\% de plantaciones de soja lo eran de variedades transgénicas, así como el 83\% del algodón y el 61\% del maíz.



\begin{table}[h!]
\centering
\begin{tabular}{|l|c|c|}
\hline
     &  Transgénicos (\%)  & No transgénicos (\%)  \\
\hline\hline
Maíz  &    58  &   42 \\
\hline
Cebada &   33   &  67   \\
\hline
Soja   &  66   & 44 \\
\hline
\multicolumn{3}{l}{\footnotesize Datos completamente inventados}
\end{tabular}
\caption{Datos sobre la producción transgénica en el año 2000}
\label{tabla.transg3}
\end{table}



%-------------------------------------------------------------------------
\section{Beneficios}



Los caracteres introducidos mediante ingeniería
genética en especies destinadas a la producción de alimentos buscan el incremento de la productividad (por ejemplo, mediante una resistencia mejorada a las plagas) así como la introducción de características de calidad nuevas. Debido al mayor desarrollo de la manipulación genética en especies vegetales, todos los alimentos transgénicos corresponden a derivados de plantas. Por ejemplo, un carácter empleado con frecuencia es la resistencia a herbicidas, puesto que de este modo es posible emplearlos afectando sólo a la flora ajena al cultivo. 



\begin{table}
\begin{center}
\begin{tabular}{|l|c|c|}
\hline
     &  Transgénicos (\%)  & No transgénicos (\%)  \\
\hline\hline
Maíz  &    58  &   42 \\
\hline
Cebada &   33   &  67   \\
\hline
Soja   &  66   & 44 \\
\hline
\multicolumn{3}{l}{\footnotesize Datos completamente inventados}
\end{tabular}
\caption{Datos sobre la producción transgénica en el año 2023}
\label{tabla.transg2}
\end{center}
\end{table}



Cabe destacar que el empleo de variedades modificadas y resistentes a herbicidas ha disminuido la contaminación debido a estos productos en acuíferos y suelo,
si bien es cierto que no se requeriría el uso de estos herbicidas tan nocivos por su alto contenido en glifosato (GLY) y amonio glifosinado (GLU)
si no se plantaran estas variedades, diseñadas exclusivamente para resistir a dichos compuestos.

\bigskip

%Como se menciona el \cite{cascales2003} y en \cite{seidel1998}

\bigskip

Las plagas de insectos son uno de los elementos más devastadores en agricultura. 
Por esta razón, la introducción de genes que provocan el desarrollo de resistentes a uno o varios órdenes de insectos ha sido un elemento común a muchas de las variedades patentadas. Las ventajas de este método suponen un menor uso de insecticidas en los campos sembrados con estas variedades,
lo que redunda en un menor impacto en el ecosistema que alberga al cultivo y por la salud de los trabajadores que manipulan los fitosanitarios.




%-------------------------------------------------------------------------
%  Entorno thebibliography
%
% Para añadir al índice general 
%
%\addcontentsline{toc}{section}{Referencias}  % si es article
%\addcontentsline{toc}{chapter}{Referencias}  % si es report o book
%-------------------------------------------------------------------------
%
%\begin{thebibliography}{99}

%
%\bibitem{cascales2003}
%Bernardo Cascales, Pascual Lucas, José~Manuel Mira, Antonio Pallarés, and
%  Salvador Sánchez-Pedreño, \emph{El libro de \LaTeX}, 2003.
%%
%\bibitem{goossens1997}
%Miochael Goossens, Sebastian Rahtz, and Frank Mittelbach, \emph{The \LaTeX{} graphics companion: illustrating documents with \TeX{} and PostScript}, Addison-Wesley Longman Publishing Co., Inc. Boston, MA, USA, 1997.
%%
%\bibitem{knuth1986}
%Donald~E. Knuth, \emph{The \TeX book}, Addison-Wesley, Reading, Mass.; Tokyo, 1986.
%%
%\bibitem{lamport1994}
%Leslie Lamport, \emph{\LaTeX: a document preparation system: user's guide and
%  reference manual}, Longman Publishing Co., Inc. Boston, MA, USA, 1994.
%%
%\bibitem{mira2004}
%José~Manuel Mira~Ros,  \emph{Bibliografía flexible: el sistema flexbib}, TeXemplares, (6):8--26, 2004.
%%
%\bibitem{seidel1998}
%Luis Seidel, \emph{Bases de datos bibliográficas, LaTeX y el idioma español}, March 1998.
%%
%\bibitem{young2002}
%David Young,  \emph{Using Bib\TeX}, May 2002.
%%
%\bibitem{young20021}
%David Young,  \emph{Using Bib\TeX}, May 2002.
%%
%\bibitem{young20022}
%David Young,  \emph{Using Bib\TeX}, May 2002.
%%
%\bibitem{young20023}
%David Young,  \emph{Using Bib\TeX}, May 2002.
%%
%\bibitem{young20024}
%David Young,  \emph{Using Bib\TeX}, May 2002.
%%
%\end{thebibliography}
%%
%================================================================


%================================================================
%
\begin{thebibliography}{123456789112345678}
%
\bibitem[\textsc{Cascales y otros}]{cascales2003}
Bernardo Cascales, Pascual Lucas, José~Manuel Mira, Antonio Pallarés, and
  Salvador Sánchez-Pedreño, \emph{El libro de \LaTeX}, 2003.
%
\bibitem[\textsc{Goossens y otros}]{Goossens1993}
Michel Goossens, Frank Mittelbach, and Alexander Samarin, \emph{The \LaTeX{} Companion}, 
Addison-Wesley Professional, 1st edition, December 1993.
%
\bibitem[\textsc{Knuth}]{knuth1986}
Donald~E. Knuth, \emph{The \TeX book}, Addison-Wesley, Reading, Mass.; Tokyo, 1986.
%
\bibitem[\textsc{Lamport}]{lamport1994}
Leslie Lamport, \emph{\LaTeX: a document preparation system: user's guide and
  reference manual}, Longman Publishing Co., Inc. Boston, MA, USA, 1994.
%
\bibitem[\textsc{Mira~Ros}]{mira2004}
José~Manuel Mira~Ros,  \emph{Bibliografía flexible: el sistema flexbib}, TeXemplares, (6):8--26, 2004.
%
\bibitem[\textsc{Seidel}]{seidel1998}
Luis Seidel, \emph{Bases de datos bibliográficas, LaTeX y el idioma español}, March 1998.
%
\bibitem[\textsc{Young}]{young2002}
David Young,  \emph{Using Bib\TeX}, May 2002.
%
\end{thebibliography}
%%
%================================================================

\tableofcontents
\listoftables


\end{document}
%%%%%%%%%%%%%%%%%%%%%%%%%%%%%%%%%%%%%%%%%%%%%%%%%%%%%%%%%

