% !TEX encoding = UTF-8 Unicode
%%%%%%%%%%%%%%%%%%%%%%%%%%%%%%%%%%%%%%%%%%
%
%  LaTeX: Composición de textos científicos con el ordenador
%              Curso de Extensión Universitaria
%                    
%%%%%%%%%%%%%%%%%%%%%%%%%%%%%%%%%%%%%%%%%%
\documentclass[10pt,a4paper]{report}
\usepackage[spanish]{babel}
\usepackage[utf8]{inputenc}
\usepackage[margin=2cm]{geometry}


%-----------------------------------------------------------------
%  paquetes necesarios para escribir matemáticas 
%
\usepackage{amsmath, amssymb, dsfont}



%-----------------------------------------------------------------
%  algunas longitudes y contadores
%  PARA COMENZAR POR EL TEMA 3
\setlength{\parindent}{0pt}
\setcounter{chapter}{2}
%-----------------------------------------------------------------

%-----------------------------------------------------------------
%  para que salga "Tema" en vez del "Capítulo"
%
\addto\captionsspanish{\def\chaptername{Tema}}


%  PAQUETE DEL CORAZÓN
\usepackage{ marvosym }

%-----------------------------------------------------------------
%  AQUÍ COMIENZA EL CUERPO
%-----------------------------------------------------------------

\begin{document}





%-----------------------------------------------------------------
% Capítulo
%-----------------------------------------------------------------
\chapter{Escribiendo matemáticas (I)}





Para escribir matemáticas se deben usar siempre los paquetes
\texttt{amsmath}, \texttt{amssymb}, \texttt{dsfont} (el último no es imprescindible).

Hay que distinguir entre dos tipos de fórmulas:

\begin{description}
%
\item[fórmulas en modo línea] 
que forman parte del texto, como en el párrafo siguiente:




\bigskip
%-----------------------------------------------------------------
Sea $f$ la función de variable real dada por $f(x) = e^x + 2x^2 + 4$, y sea $\alpha = f(-1)$
%-----------------------------------------------------------------
\bigskip


%
\item[fórmulas centradas o resaltadas (displayed)]
como la siguiente, que se consigue usando el entorno \texttt{equation}




\bigskip
%-----------------------------------------------------------------
Sea $f$ la función de variable real dada por

\begin{equation}\label{eq1}
f(x) = e^x + 2x^2 + 4
\end{equation}

%-----------------------------------------------------------------
\bigskip





Si no se desea numerar la fórmula se usa el entorno \texttt{equation*}

\begin{equation*}
f(x) = e^x + 2x^2 + 4
\end{equation*}

\bigskip
%-----------------------------------------------------------------
xxxxxxxxxxxxxxxxxxxx
%-----------------------------------------------------------------
\bigskip



%
\end{description}





A las fórmulas centradas numeradas se les puede añadir una etiqueta y pueden ser referenciadas
mediante los comandos habituales (\verb+\ref{}+, \verb+\pageref{}+) y además \verb+\eqref{}+.



\bigskip
%-----------------------------------------------------------------
\textbf{Ejemplo}

La fórmula (\ref{eq1}) está en la página \pageref{eq1}
%-----------------------------------------------------------------
\bigskip






%-----------------------------------------------------------------
% sección
%-----------------------------------------------------------------
\section{Elementos básicos}




Los siguientes símbolos se escriben tal cual:
\begin{equation*}
[ ( + - = / | < > : ? ! ) ]
\end{equation*}
así como las letras del alfabeto y los números. Las llaves se consiguen con \verb+\{+ y \verb+\}+.
Las funciones y las letras griegas se escriben mediante comandos:





\bigskip
%-----------------------------------------------------------------
\textbf{Ejemplos}

\begin{equation*}
y'(x)=\cos(x)-(f(x)+\alpha \beta /n!)
\end{equation*}

\begin{equation*}
Q=\{(x,y) : x^2+y^2<1\}
\end{equation*}

%-----------------------------------------------------------------
\bigskip





Los símbolos matemáticos tienen en general nombres bastante obvios o fáciles de recordar (en inglés):




\bigskip
%-----------------------------------------------------------------
\textbf{Ejemplos}

\begin{equation*}
x \in [a,b], \quad x \to \infty, \quad \forall \varepsilon >0 \quad \exists\, \delta >0
\end{equation*}


\begin{equation*}
A \subset \mathds{R}, \quad (A \cup B)\cap C,\quad, a<b \Longrightarrow f(a)<f(b)
\end{equation*}
\Heart

%-----------------------------------------------------------------
\bigskip





Dentro de un entorno \texttt{equation} no puede haber líneas en blanco. 

Para escribir texto normal dentro de una fórmula se debe usar el comando \verb+\text+




\bigskip
%-----------------------------------------------------------------
\textbf{Ejemplo}

\begin{equation*}
f(x)=0 \text{ para casi todo } x \in [0,M]
\end{equation*}

\begin{equation*}
\text{Se considera }F(v)=B(u,v)\text{ con }v \in X
\end{equation*}
%-----------------------------------------------------------------
\bigskip






%-----------------------------------------------------------------
\subsection{Exponentes y subíndices}



Para escribir exponentes y subíndices se utilizan, respectivamente, 
los comandos \verb+^+ y \verb+_+
que se aplican sólo al carácter que les sigue. Para que se apliquen a más de un carácter hay que agruparlos usando llaves.



\bigskip
%-----------------------------------------------------------------
\textbf{Ejemplo}

\begin{equation*}
x^2, \quad x_k, \quad y^\alpha, \quad y_j^\alpha, \quad, 10^{-7}, \quad e^{x^2+1}, \quad A=(a_{ij})^N_{i,j=1}, \quad C_{i_j}^{k^{r^t}}
\end{equation*}
%-----------------------------------------------------------------
\bigskip





%-----------------------------------------------------------------
\subsection{Fracciones}



Las {\bf fracciones} se componen usando el comando
\verb+\frac{numerador}{denominador}+


\bigskip
%-----------------------------------------------------------------
\textbf{Ejemplo}

Una fracción en línea como esta $\frac{x^2}{1+a}$ se escribe diferente de esta:

\begin{equation*}
\frac{x^2}{1+a},
\end{equation*}

Una fracción en línea como esta $\displaystyle\frac{x^2}{1+a}$ se escribe diferente de esta:

Una fracción en línea como esta $\dfrac{x^2}{1+a}$ se escribe diferente de esta:
%-----------------------------------------------------------------
\bigskip




\bigskip
%-----------------------------------------------------------------
\textbf{Ejemplo}
\begin{equation*}
\frac{y}{(x+z)^2},
\quad
\frac{1}{2}\frac{x^3}{n!},
\quad
\frac{\frac{a+b}{c^4}}{(c-e+1)},
\quad
\frac{x}{\frac{1}{y}+\frac{z^2+1}{3}}.
\end{equation*}



%-----------------------------------------------------------------
\bigskip




El comando
\verb+\displaystyle+ 
sirve para forzar a que determinadas construcciones, como por ejemplo fracciones,
se compongan, en modo de línea, en el mismo tamaño que se compondrían en modo resaltado.
Para las fracciones en particular se puede usar
\verb+\dfrac{numerador}{denominador}+



\bigskip
%-----------------------------------------------------------------
\textbf{Ejemplo}

xxxxxxxxxxxxxxxxxxxx

xxxxxxxxxxxxxxxxxxxx
%-----------------------------------------------------------------
\bigskip




%-----------------------------------------------------------------
\subsection{Raíces y números binomiales}



\bigskip
%-----------------------------------------------------------------
\textbf{Ejemplo}

\begin{equation*}
\sqrt{x-y-z},
\quad
\sqrt[n]{\frac{x^3+y^4}{(x+y)^5}},
\quad
\sqrt{2+\sqrt{2+\sqrt{2}}},
\quad
\binom{a}{b}
\end{equation*}
%-----------------------------------------------------------------
\bigskip






%-----------------------------------------------------------------
\subsection{Sumatorios e integrales}



Se consiguen con los comandos \verb+\int_{límite inferior}^{límite superior}+ y  \newline
\verb+\sum_{límite inferior}^{límite superior}+



\bigskip
%-----------------------------------------------------------------
\textbf{Ejemplo}

$\int$, \quad $\displaystyle\int$\quad $\sum$ \quad $\Sigma$

\bigskip

Un sumatorio en modo párrafo es:
%
$\sum_{n=0}^{\infty}(-1)^n\frac{x^{2n+1}}{n!(2n+1)}
=
\int_{0}^{x}e^{-t^2}\,dt$ 
y con el display style: 
$\displaystyle\sum_{n=0}^{\infty}(-1)^n\frac{x^{2n+1}}{n!(2n+1)}
=
\int_{0}^{x}e^{-t^2}\,dt$. Con el modo equation:

\begin{equation}
\sum_{n=0}^{\infty}(-1)^n\frac{x^{2n+1}}{n!(2n+1)}
=
\int_{0}^{x}e^{-t^2}\,dt
\end{equation}

%-----------------------------------------------------------------
\bigskip




\bigskip
%-----------------------------------------------------------------
\textbf{Ejemplo}

\begin{equation}
\sum_{n=0}^{\infty}(-1)^nx^{3n},
\quad
\sqrt[7]{\sum_{i=1}^{m}a_i^jy^2},
\quad
\int_{1}^{y^2}\frac{2+x}{x^3+1}
\end{equation}
%-----------------------------------------------------------------
\bigskip




%-----------------------------------------------------------------
\subsection{Otros operadores con límites}





\bigskip
%-----------------------------------------------------------------
\textbf{Ejemplo}

Modo párrafo: $\sup_{x\in\Omega}$
\begin{equation}
\sup_{x\in\Omega} |f(x)|,
\quad
\lim_{n\to\infty}y_n=\alpha,
\quad
z=\max_{x\in\mathbb{R}}\frac{f(x)}{x}
\end{equation}
%-----------------------------------------------------------------
\bigskip








%-----------------------------------------------------------------
\subsection{Puntos suspensivos}




\bigskip
%-----------------------------------------------------------------
\textbf{Ejemplo}
\begin{equation}
\{x_1,\dots,x_n\}
\quad
x1 + \cdots + x_n,
\quad
\ddots,
\quad
\vdots
\end{equation}

%-----------------------------------------------------------------
\bigskip





%-----------------------------------------------------------------
\subsection{Paréntesis de todos los tamaños}



\bigskip
%-----------------------------------------------------------------
\textbf{De tamaño fijo}

\begin{equation}
(a,b],
\quad
\big( (a,b] \big],
\quad
\Big(\big( (a,b] \big]\Big],
\quad
\bigg(\Big(\big( (a,b] \big]\Big]\bigg],
\quad
\Bigg(\bigg(\Big(\big( (a,b] \big]\Big]\bigg]\Bigg]
\end{equation}
%-----------------------------------------------------------------
\bigskip






\bigskip
%-----------------------------------------------------------------
\textbf{De tamaño adaptativo}

\begin{equation}
\left(\frac{x}{\frac{x^2+1}{e^{x^3}}}\right\},
\quad
\left[\int_{0}^{x^2}f(x)\,dx\right.
\end{equation}
%-----------------------------------------------------------------
\bigskip







%-----------------------------------------------------------------
\end{document}
%-----------------------------------------------------------------



