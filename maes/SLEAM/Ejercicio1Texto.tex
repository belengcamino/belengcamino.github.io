% !TEX encoding = UTF-8 Unicode
%------------------------------------------------------------------
% Curso : Edición de textos científicos con ordenador
%------------------------------------------------------------------
% EJERCICIO 1: Resumen de la primera clase
%------------------------------------------------------------------
\documentclass[a4paper]{report}
\usepackage[spanish]{babel}
\usepackage[utf8]{inputenc} % Para los acentos




%------------------------------------------------------------------
%  AQUÍ COMIENZA EL CUERPO
%------------------------------------------------------------------
\begin{document}





%------------------------------------------------------------------
%  chapter
%------------------------------------------------------------------
\chapter{Algunos aspectos generales}




%------------------------------------------------------------------
%  section
%------------------------------------------------------------------
\section{Lo más básico: la clase de documento}

El fichero fuente de \LaTeX{}
contiene \textit{comandos} y \textit{textos}. 
Es obligatorio que tenga \textbf{preámbulo} y \textbf{cuerpo}.

El aspecto global del documento está determinado por la clase del
mismo. Algunas de las clases nativas de \LaTeX{} son: 

\noindent
\texttt{report}, para \textit{memorias}, informes, proyectos de fin de carrera,
tesis doctorales, apuntes, \dots (este ejercicio utiliza la clase \texttt{report}); 

\noindent
\texttt{book}, para \textit{libros}; 

\noindent
\texttt{article}, para \textit{artículos}, o documentos no muy grandes; 

\noindent
\texttt{letter}, para escribir \textit{cartas}.




%------------------------------------------------------------------
%  section
%------------------------------------------------------------------
\section{Los espacios}


Cuando se escribe un texto en \LaTeX{}, no es necesario estar pendiente 
de cuándo terminar una línea y empezar otra, esto se hace de forma automática.

Para iniciar un nuevo párrafo basta dejar una línea en blanco en el código fuente. 
El efecto de varias líneas en blanco es el mismo que él una sola línea.

Para conseguir una separación mayor, como aquí \hspace{3cm} hay que usar los
comandos correspondientes. En la línea anterior se ha usado un espacio horizontal de 3 cm. En el párrafo siguiente se ha usado un salto vertical equivalente a dejar una línea en blanco.\bigskip

Existen, además, órdenes que introducen saltos de longitud ``variable'', 
es decir, que se adaptan al espacio disponible. 
Normalmente se utilizan para introducir espacio entre dos elementos 
``empujándolos'' (a izquierda y derecha, o bien arriba y abajo) hasta donde sea posible (como muestra el texto que sigue). 

Izquierda \hfill derecha.

\vfill

\noindent
Este párrafo no está sangrado --sí lo está el párrafo anterior--. El texto de este párrafo está ``empujado'' hacia el pie de página.


\end{document}

%%%%%%%%%%%%%%%%%%%%%%%%%%%%%%%%%%%%%%%%%%%%%%%%%%%%%%%%%%%%%%%%%%%
%%%%%%%%%%%%%%%%%%%%%%%%%%%%%%%%%%%%%%%%%%%%%%%%%%%%%%%%%%%%%%%%%%%
%%%%%%%%%%%%%%%%%%%%%%%%%%%%%%%%%%%%%%%%%%%%%%%%%%%%%%%%%%%%%%%%%%%
%%%%%%%%%%%%%%%%%%%%%%%%%%%%%%%%%%%%%%%%%%%%%%%%%%%%%%%%%%%%%%%%%%%



