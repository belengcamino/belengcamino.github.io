% !TEX encoding = UTF-8 Unicode
%--------------------------------------------------------------------
% Curso : Edición de textos científicos con LaTeX
%--------------------------------------------------------------------
% Fichero de trabajo para tablas
%--------------------------------------------------------------------

\documentclass[11pt, a4paper]{article}
\usepackage[utf8]{inputenc}
\usepackage[spanish]{babel}
\usepackage{color}







%%%%%%%%%%%%%%%%%%%%%%%%%%%%%%%%%%%%%%%%%%%%%%%%%%%%%%%%%%%%%%%%%%%%%
\begin{document}





%--------------------------------------------------------------------
% Sección 
%--------------------------------------------------------------------
\section{El entorno tabular}






\bigskip
%--------------------------------------------
% tabla simple
%
\begin{tabular}{lcl}
1 & 2 & 3 \\ 
4 & 5 & 6 \\ 
7 & 8 & 9 
\end{tabular}  




\bigskip
%--------------------------------------------
% tabla simple con rayas
%
\begin{tabular}{|l|c||l|} 
\hline 
1 & 2 & 3 \\ 
\hline\hline
4 & 5 & 6 \\ 
7 & 8 & 9 \\
\hline
\end{tabular}




\bigskip
%--------------------------------------------
% rayas que se extiende sólo a algunas columnas
% \cline{n-m}
%
\begin{tabular}{|l|c||l|} 
\hline 
1 & 2 & 3 \\ 
\hline\hline
4 & 5 & 6 \\ 
\cline{2-3}
7 & 8 & 9  \\
\cline{1-2}
\end{tabular}


\begin{tabular}{|l|c||l|} 
\hline 
1 & 2 & 3 \\ 
\hline\hline
4 & 5 & 6 \\ 
\cline{3-3} % NO COGE LAS DOS LINEAS VERTICALES PORQUE PERTENECEN A LA SEGUNDA COLUMNA
7 & 8 & 9  \\
\cline{1-2}
\end{tabular}








\bigskip
%---------------------------------------------
% LaTeX no restringe la anchura de las columnas
%
\begin{tabular}{|l|c||l|}
\hline
1 & 2 & bla bla bla bla bla bla bla bla bla bla bla bla bla bla bla bla bla bla bla bla  bla bla bla bla bla bla \\
\hline\hline
4 & 5 & bla bla bla bla bla bla bla bla bla bla bla  \\ 
7 & 8 & bla bla bla bla bla bla bla bla bla bla bla  \\ 
\hline
\end{tabular} 









\bigskip
%---------------------------------------------
% la opcion p{} sirve para componer una columna 
% en modo párrafo
%
\begin{tabular}{|l|c||p{3cm}|}
\hline
1 & 2 & bla bla bla bla bla bla bla bla bla bla bla  \\
\hline\hline
4 & 5 & bla bla bla bla bla bla bla bla bla bla bla  \\ 
7 & 8 & bla bla bla bla bla bla bla bla bla bla bla  \\ 
\hline
\end{tabular} 





\bigskip
%---------------------------------------------
% separador de columnas @{algo} suprime el espacio 
% entre columnas e inserta "algo" entre ellas
%
\begin{tabular}{|l@{ dijo: }p{2.5cm}|}
\hline
Isabel     &  bla bla bla bla bla bla bla bla bla bla  \\
\hline
Paulina    &  bla bla bla bla bla bla bla bla bla bla  \\ 
\hline
Macarena &  bla bla bla bla bla bla bla bla bla bla  \\
\hline
\end{tabular} 







\bigskip
%---------------------------------------------
% tabla de 3 columnas: no hay espacio entre la 
% coma decimal y los números
%
\begin{tabular}{|l|r@{,}l|}
\hline
Fulanito     &   5 & 1 \\
\hline
Menganito    &   7 & 25 \\
\hline
Zutanito     &   6 & 75 \\
\hline
\end{tabular} 




\bigskip
%---------------------------------------------
% @{} permite crear separaciones entre columnas
%
\begin{tabular}{|l|@{\hspace{20mm}}r@{,}l@{\hspace{5mm}}|}
\hline
Fulanito     &   5 & 1 \\
\hline
Menganito    &   7 & 25 \\
\hline
Zutanito     &   6 & 75 \\
\hline
\end{tabular} 






\bigskip
%---------------------------------------------
% @{} permite suprimir espacios entre columnas 
% aunque no se incluya otra cosa
%
\begin{tabular}{|r@{}c@{}l|}
\hline
1 & 2 & 3 \\ 
\hline\hline
44 & 5 & 6 \\ 
7 & 8 & 93 \\
\hline
\end{tabular}


















%--------------------------------------------------------------------
% Sección 
%--------------------------------------------------------------------
\section{Celdas que se expanden a varias columnas}

%------------------------------------------------
%
% \multicolumn{numero}{formato}{contenido}
% 
%  número: es el número de columnas a las que se expande la celda
%  formato: es la descripción de la alineación en la celda (como en tabular)
%  contenido: es el texto que se coloca en la celda
%
%------------------------------------------------







\bigskip
%---------------------------------------------
% uso de multicolumn
%
\begin{tabular}{|l|l|}
\hline
\multicolumn{2}{|c|}{Cosas raras y rojas} \\
\hline\hline
celda \textcolor{red}{roja} & tabla \textbf{rara}\\
celda \textcolor{red}{roja} & tabla \textbf{rara}\\
\hline
\end{tabular}







\bigskip
%---------------------------------------------
% uso de multicolumn
%
\begin{tabular}{|c|c|c|}
\hline
\multicolumn{3}{|c|}{Cosas raras y rojas}          \\
\hline\hline
\textcolor{red}{rojas} & \multicolumn{2}{c|}{raras}       \\
\hline\hline
celda \textcolor{red}{roja} & tabla \textbf{rara} & tabla \textbf{rara}\\
\cline{2-3}
celda \textcolor{red}{roja} & tabla \textbf{rara} & tabla \textbf{rara}\\
\hline
\end{tabular}





\bigskip
%--------------------------------------------------------------------
% Sección 
%--------------------------------------------------------------------
\section{Alineación vertical de las tablas}






\bigskip
%---------------------------------------------
% alineación vertical
%
Queremos ver cómo la tabla  
\begin{tabular}{|c|c|} 
\hline A & B \\ \hline C & D \\ \hline
\end{tabular} 
se alinea con el texto \dots






\bigskip
%---------------------------------------------
% alineación vertical
%
Queremos ver cómo la tabla  
\begin{tabular}[b]{|c|c|} 
\hline A & B \\ \hline C & D \\ \hline
\end{tabular} 
se alinea con el texto \dots







\bigskip

%---------------------------------------------
% alineación vertical
%
Queremos ver cómo la tabla  
\begin{tabular}[t]{|c|c|} 
\hline A & B \\ \hline C & D \\ \hline
\end{tabular} 
se alinea con el texto \dots



\bigskip
%Si quitamos la primera línea horizontal
Queremos ver cómo la tabla  
\begin{tabular}[t]{|c|c|} 
A & B \\ \hline C & D \\ \hline
\end{tabular} 
se alinea con el texto \dots



\newpage
%%--------------------------------------------------------------------
%% Sección 
%%--------------------------------------------------------------------
\section{Solución del Ejercicio 4.24 de los apuntes}
%
%
%
\bigskip
%%---------------------------------------------
%% tabla del Ejercicio 4.24
%%
\begin{center}
\textbf{\Large Tabla del Ejercicio 4.24}
\vskip 1cm

\begin{tabular}{|l|r@{,}l|r@{,}l|r@{--}l@{horas\quad}|p{5cm}|}
%%
\hline
\multicolumn{8}{|c|}{\textbf{Tres alumnos}}
\\ 
%%
\hline \hline
    & \multicolumn{4}{c|}{Nota Media} 
    & \multicolumn{2}{c|}{\begin{tabular}{c} Estudia  \\
                                             diariamente
                          \end{tabular}} 
    &
\\
%%
\cline{2-7}
   Nombre 
 & \multicolumn{2}{c|}{BUP} 
 & \multicolumn{2}{c|}{COU} 
 & \multicolumn{1}{r|}{mínimo} 
 & \multicolumn{1}{l|}{máximo} 
 & \multicolumn{1}{c|}{Observaciones}
\\
%%
\hline
Pedro & 6 & 75 & 6 & 5 & 2 & 3 & Es un buen estudiante, aunque a
                                 veces se distrae con facilidad.
\\
%%
\hline
Javier & 8 & 5 & 7 & 75 & 1 & 3 & Muy buen estudiante. Presta mucha
                                  atención en clase y rara vez está
                                  distraído.
\\
%%
\hline
Adolfo & 5 & 25 & 5 & 5 & \multicolumn{2}{c|}{desconocido} &
                                  Demuestra ser inteligente, 
                                  pero le falta la concentración.
\\
%%
\hline
\end{tabular}
\end{center}
 



\section{Solución del Ejercicio 4.21 de los apuntes}
\begin{center}
\textbf{\Large Tabla del Ejercicio 4.24}
\vskip 1cm
\begin{tabular}{|c|c|c|c|}
\hline
\textsc{País} & \textsc{Capital} & \multicolumn{2}{c|}{\textsc{Poblacion y Superficie}} \\
\hline
España & Madrid & 37.746.260 hab. & 504.750 Km$^2$\\
\cline{2-3}
Francia & París & 55.191.000 hab. & 543.998 Km$^2$ \\
\cline{2-2}\cline{4-4}
N & X & no se conoce* & pocos KM$^2$ \\
\hline
\multicolumn{4}{l}{* se está estudiando}
\end{tabular}

\end{center}

\end{document}
%%%%%%%%%%%%%%%%%%%%%%%%%%%%%%%%%%%%%%%%%%%%%%%%%%%%%%%%%%%%%%%%%%%%%

