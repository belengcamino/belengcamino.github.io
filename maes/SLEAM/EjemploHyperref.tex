% !TEX encoding = UTF-8 Unicode
%%%%%%%%%%%%%%%%%%%%%%%%%%%%%%%%%%%%%%%%%%%%%%%%%%%%%%%%%%%%%%%%%%%%%%%%%%
%  Ejemplos de uso del paquete hyperref que permite transformar en
%  hiper-enlaces las referencias cruzadas de un documento,
%  incluyendo las citas bibliográficas.
%  También permite incluir hiper-enlaces a documentos y URL's externas  
%-------------------------------------------------------------------------
%  La simple declaración en el documento del paquete
%  \usepackage[opciones]{hyperref}
%  transforma en hiper-enlaces todas las referencias cruzadas
%
%  Algunas opciones de hyperref:
%
%  colorlinks  :  true  (hiperenlaces coloreados)
%  linkcolor   :  nombre de un color
%  citecolor   :  nombre de un color
%  urlcolor    :  nombre de un color
%
%-------------------------------------------------------------------------
% Para especificar opciones alternativamente se puede usar el comando
%  \hypersetup{opcion1=valor1,opcion2=valor2,...}. Por ejemplo, 
%  \hypersetup{colorlinks=true}
%
%-------------------------------------------------------------------------
%  ATENCIÓN: el paquete hyperref debe ser el último en ser cargado
%-------------------------------------------------------------------------
% 
%  Comando básico: 
%  \href{Tipo:URL}{texto}
%
%   Tipo:
%       http      : por defecto, no hace falta indicarlo
%       mailto  : para enviar un correo electrónico
%       fpt        : para conectar con servidores FTP 
%.      file       : abre un archivo con el programa asociado
%       run      : permite ejecutar una aplicación 
%
%-------------------------------------------------------------------------
%  Otros comandos:
%
%  \url{URL}
%  \hyperref[etiqueta]{texto}
%  \hypertarget{etiqueta}{texto a mostrar} 
%  \hyperlink{etiqueta}{texto de enlace}
%
%-------------------------------------------------------------------------

\documentclass[11pt,a4paper]{article}
\usepackage[spanish]{babel}
\usepackage[utf8]{inputenc}
\usepackage{amsmath,amssymb} 
%
\usepackage{graphicx}
\usepackage[dvipsnames,usenames]{color}
\usepackage{wrapfig}

%
%-------------------------------------------------------------------------
% Descomentar uno de los comandos que siguen:
%------------------------------------------------------------------------
%\usepackage{hyperref}
%\usepackage[colorlinks=true]{hyperref}
% \usepackage[colorlinks=true, linkcolor=OrangeRed, citecolor=red,]{hyperref}
 \usepackage[colorlinks=true,linkcolor=OrangeRed,citecolor=blue,urlcolor=OliveGreen]{hyperref}
%

%-------------------------------------------------------------------------
\addto\captionsspanish{\def\tablename{Tabla}}
\addto\captionsspanish{\renewcommand{\listtablename}{Índice de tablas}}





%-------------------------------------------------------------------------
%  Poner enlaces usando \href
%
\title{Alimentos transgénicos
\thanks{Resumen del artículo publicado en la revista Science - 
\href{http://www.sciencemag.org/}{http://www.sciencemag.org/}}}
%
\author{Perico Palotes
\thanks{\href{mailto:perico.palotes@us.es}{Pincha aquí para mandar el email}}}
%
\date{Departamento de Tecnología Alimentaria
\thanks{\url{http://www.sciencemag.org/}}}
%-------------------------------------------------------------------------





%%%%%%%%%%%%%%%%%%%%%%%%%%%%%%%%%%%%%%%%%%%%%%%%%%%%%%%%%%%%%%%%%%%%%%%%%%
\begin{document}


\maketitle


\tableofcontents

\newpage 
%-------------------------------------------------------------------------


\parskip5mm
%%%%%%%%%%%%%%%%%%%%%%%%%%%%%%%%%%%%%%%%%%%%%%%%%%%%%%%%%%%%%%%%%%%%%%%%%%
\section{Introducción}

Los alimentos sometidos a ingeniería genética o alimentos transgénicos son aquellos que fueron producidos a partir de un organismo modificado genéticamente mediante ingeniería genética. Dicho de otra forma, es aquel alimento obtenido de un organismo al cual le han incorporado genes de otro para producir una característica deseada. En la actualidad tienen mayor presencia alimentos procedentes de plantas transgénicas como el maíz, la cebada o la soja 




\begin{table}[h!]
\centering
\begin{tabular}{|l|c|c|}
\hline
     &  Transgénicos (\%)  & No transgénicos (\%)  \\
\hline\hline
Maíz  &    58  &   42 \\
\hline
Cebada &   33   &  67   \\
\hline
Soja   &  66   & 44 \\
\hline
\multicolumn{3}{l}{\footnotesize Datos completamente inventados}
\end{tabular}
\caption{Datos sobre la producción transgénica en el año 1500}
\label{tabla.transg}
\end{table}



La ingeniería genética o tecnología del ADN recombinante es la ciencia que manipula secuencias de ADN ---que normalmente codifican genes--- de forma directa, posibilitando su extracción de un taxón biológico dado y su inclusión en otro, así como la modificación o eliminación de estos genes. En esto se diferencia de la mejora clásica, que es la ciencia que introduce fragmentos de ADN (conteniendo como en el caso anterior genes) de forma indirecta, mediante cruzamientos dirigidos.\cite{brezis} 

\begin{equation}\label{edo3}
\varphi'(t)\left(=\frac{d\varphi}{dt}(t)\right)=f(t,\varphi(t)),\quad\forall\, t\in I.
\end{equation}




La primera estrategia, la de la ingeniería genética, se circunscribe en la disciplina denominada biotecnología vegetal
%-----------------------------------------------------------------
%
%   Enlace a una etiqueta del texto: \hyperref[etiqueta]{texto} 
%
%-----------------------------------------------------------------
\hyperref[seccion.beneficios]{véase la Sección~\ref{seccion.beneficios}}.


Cabe destacar que la inserción de grupos de genes mediante obtención de híbridos ---incluso de especies distintas--- y otros procesos pueden realizarse mediante técnicas de biotecnología vegetal que no son consideradas ingeniería genética, como puede ser la fusión de protoplastos~\cite{padula1}
%------------------------------------------------------
%
%   Enlace a una etiqueta del texto: \hyperref[etiqueta]{texto} 
%
%---------------------------- --------------------------
\hyperref[eq4]{véase la ecuación~(\ref{eq4})}.




\begin{equation}\label{eq1}
\int\!\!\!\int_Q\rho^2|\varphi|^2\,dx\,dt
\le
\int\!\!\!\int_\omega\rho^2|\varphi_t-\Delta\varphi|^2\,dx\,dt
\end{equation}



La mejora de las especies que serán usadas como alimento ha sido un motivo común en la historia de la Humanidad.
 Entre el 12000 y 4000 a. de C. ya se realizaba una mejora por selección artificial de plantas (véase la ecuación~(\ref{eq1})) Tras el descubrimiento de la reproducción sexual en vegetales, se realizó el primer cruzamiento intergenérico (es decir, entre especies de géneros distintos) en 1876. En 1909 se efectuó la primera fusión de protoplastos, y en 1927 se obtuvieron mutantes de mayor productividad mediante irradiación con rayos X de semillas.
Finalmente, en 1983 se produjo la primera planta transgénica y en 1994 se aprobó la comercialización del primer alimento modificado genéticamente.\cite{simon3}

%------------------------------------------------------------------
%
% Enlace a la palabra "plagas": \hyperlink{etiqueta}{texto de enlace}
%
%------------------------------------------------------------------

\hyperlink{target.plagas}{Este enlace lleva a la palabra plagas}


\begin{figure}[h]
\centering
\includegraphics[width=10cm,draft=false]{images/mapamundo}
\caption[Producción mundial de GMO]{Áreas con cultivos de GMO ---Genetically Modified Organism--- en el año 2005. Los cinco países marcados en color naranja produjeron más 
del 95\% de GMO}
\label{mapamundo}
\end{figure}


En el año 2007, los cultivos de transgénicos se extienden en 114,3 millones de hectáreas de 23 países, de los cuales 12 son países en vías de desarrollo.\cite{enrique}
 En el año 2006 en Estados Unidos el 89\% de plantaciones de soja lo eran de variedades transgénicas, así como el 83\% del algodón y el 61\% del maíz.\cite{chia}



\begin{table}
\centering
\begin{tabular}{|l|c|c|}
\hline
     &  Transgénicos (\%)  & No transgénicos (\%)  \\
\hline\hline
Maíz  &    58  &   42 \\
\hline
Cebada &   33   &  67   \\
\hline
Soja   &  66   & 44 \\
\hline
\multicolumn{3}{l}{\footnotesize Datos completamente inventados}
\end{tabular}
\caption{Datos sobre la producción transgénica en el año 2000}
\label{tabla.transg3}
\end{table}




%%%%%%%%%%%%%%%%%%%%%%%%%%%%%%%%%%%%%%%%%%%%%%%%%%%%%%%%%%%%%%%%%%%%%%%%%%
\section{Beneficios}\label{seccion.beneficios}



Los caracteres introducidos mediante ingeniería genética en especies destinadas a la producción de alimentos buscan el incremento de la productividad (por ejemplo, mediante una resistencia mejorada a las plagas) así como la introducción de características de calidad nuevas. Debido al mayor desarrollo de la manipulación genética en especies vegetales, todos los alimentos transgénicos corresponden a derivados de plantas. Por ejemplo, un carácter empleado con frecuencia es la resistencia a herbicidas, puesto que de este modo es posible emplearlos afectando sólo a la flora ajena al cultivo. Cabe destacar que el empleo de variedades modificadas y resistentes a herbicidas ha disminuido la contaminación debido a estos productos en acuíferos y suelo,\cite{kisko}
si bien es cierto que no se requeriría el uso de estos herbicidas tan nocivos por su alto contenido en glifosato (GLY) y amonio glifosinado (GLU) \cite{amrouche} 
si no se plantaran estas variedades, diseñadas exclusivamente para resistir a dichos compuestos. \cite{frank}



\begin{table}
\begin{center}
\begin{tabular}{|l|c|c|}
\hline
     &  Transgénicos (\%)  & No transgénicos (\%)  \\
\hline\hline
Maíz  &    58  &   42 \\
\hline
Cebada &   33   &  67   \\
\hline
Soja   &  66   & 44 \\
\hline
\multicolumn{3}{l}{\footnotesize Datos completamente inventados}
\end{tabular}
\caption{Datos sobre la producción transgénica en el año 2023}
\label{tabla.transg2}
\end{center}
\end{table}


%
\begin{equation}\label{eq2}
\left\{\begin{array}{l}   
	y'=\dfrac{y^2-1}{t}  \\  
	y(1)=1/2 
\end{array}\right.
\end{equation}
%

%-----------------------------------------------------------------
%
%  Target (objetivo de llegada) la palabra "plagas":
% 
%  \hypertarget{etiqueta}{texto de llegada}
% 
%-----------------------------------------------------------------

Las \hypertarget{target.plagas}{\textbf{plagas}} de insectos son uno de los elementos más devastadores en agricultura.
Por esta razón, la introducción de genes que provocan el desarrollo de resistentes a uno o varios órdenes de insectos ha sido un elemento común a muchas de las variedades patentadas. Las ventajas de este método suponen un menor uso de insecticidas en los campos sembrados con estas variedades,
lo que redunda en un menor impacto en el ecosistema que alberga al cultivo y por la salud de los trabajadores que manipulan los fitosanitarios~\cite{olga1}.


%-----------------------------------------------------------------
%
%  Enlace al target "tabaco": pinchando en la imagen ciruela 
%                             queremos enlazar con la imagen tabaco
%
%-----------------------------------------------------------------

\begin{figure}[h]
\centering
\hyperlink{target.tabaco}{\includegraphics[width=6cm]{images/ciruela}}
\caption{Ciruela transgénica}
\label{ciruela}
\end{figure}




%%%%%%%%%%%%%%%%%%%%%%%%%%%%%%%%%%%%%%%%%%%%%%%%%%%%%%%%%%%%%%%%%%%%%%%%%%
\section{Polémica}\label{seccion.polemica}




En varios países del mundo han surgido grupos opuestos a los organismos genéticamente modificados, formados principalmente por ecologistas, asociaciones de derechos del consumidor, algunos científicos y políticos, los cuales exigen el etiquetaje de estos, por sus preocupaciones sobre seguridad alimentaria, impactos ambientales, cambios culturales y dependencias económicas. Llaman a evitar este tipo de alimentos, cuya producción involucraría daños a la salud, ambientales, económicos, sociales y problemas legales y éticos por concepto de patentes.
De este modo, surge la polémica derivada entre sopesar las ventajas e inconvenientes del proceso. Es decir: el impacto beneficioso en cuanto a economía, 
estado medioambiental del ecosistema aledaño al cultivo 
y en la salud del agricultor ha sido descrito, \cite{kim}
pero las dudas respecto a la posible aparición de alergias, 
cambios en el perfil nutricional, dilución del acervo genético y difusión de resistencias a antibióticos también.



\begin{figure}[h]
\centering
\includegraphics[width=8cm]{images/protesta}
\caption[Manifestación pro agricultura ecológica]{Protesta de organizaciones agrarias españolas en contra de los transgénicos en la agricultura ecológica (Puerta del Sol de Madrid, 30 de agosto de 2008).}
\label{protesta}
\end{figure}


\begin{equation}\label{eq3}
\int\!\!\!\int_Q\rho^2|\varphi|^2\,dx\,dt
\le
\int\!\!\!\int_\omega\rho^2|\varphi_t-\Delta\varphi|^2\,dx\,dt
\end{equation}

%%%%%%%%%%%%%%%%%%%%%%%%%%%%%%%%%%%%%%
\subsection{Transferencia horizontal}


Se ha postulado el papel de los alimentos transgénicos en la difusión de la resistencia a antibióticos, pues la inserción de ADN foráneo en las variedades transgénicas puede hacerse (y en la mayoría de los casos se hace) mediante la inserción de marcadores de resistencia a antibióticos.
No obstante, se han desarrollado alternativas para no emplear este tipo de genes o para eliminarlos de forma limpia de la variedad final 
y, desde 1998, la FDA exige que la industria genere este tipo de plantas sin marcadores en el producto final. \cite{okamoto}
La preocupación por tanto es la posible transferencia horizontal de estos genes de resistencia a otras especies, como bacterias de la microbiota del suelo (rizosfera) o de la microbiota intestinal de mamíferos (como los humanos). Teóricamente, este proceso podría llevarse a cabo por transducción, conjugación y transformación, si bien esta última (mediada por ADN libre en el medio) parece el fenómeno más probable. Se ha postulado, por tanto, que el empleo de transgénicos podría dar lugar a la aparición de resistencias a bacterias patógenas de relevancia clínica. 



\begin{equation}\label{eq4}
\int\!\!\!\int_Q\rho^2|\varphi|^2\,dx\,dt
\le
\int\!\!\!\int_\omega\rho^2|\varphi_t-\Delta\varphi|^2\,dx\,dt
\end{equation}



Sin embargo, existen multitud de elementos que limitan la transferencia de ADN del producto transgénico a otros organismos. El simple procesado de los alimentos durante previo al consumo degrada el ADN.
Además, en el caso particular de la transferencia de marcadores de resistencia a antibióticos, las bacterias del medio ambiente poseen enzimas de restricción que degradan el ADN que podría transformarlas (este es un mecanismo que emplean para mantener su estabilidad genética).
Más aún, en el caso de que el ADN pudiera introducirse sin haber sido degradado en los pasos de procesado de alimentos y durante la propia digestión, debería recombinarse de forma definitiva en su propio material genético, lo que, para un fragmento lineal de ADN procedente de una planta requiriría una homología de secuencia muy alta, o bien la formación de un replicón independiente.
No obstante, se ha citado la penetración de ADN intacto en el torrente sanguíneo de ratones que habían ingerido un tipo de ADN denominado M13 ADN que puede estar en las construcciones de transgénicas, e incluso su paso a través de la barrera placentaria a la descendencia.
En cuanto a la degradación gastrointestinal, se ha demostrado que el gen epsps de soja transgénica sigue intacto en el intestino.
Por tanto, puesto que se ha determinado la presencia de algunos tipos de ADN transgénico en el intestino de mamíferos, debe tenerse en cuenta la posibilidad de una integración en el genoma de la microbiota intestinal (es decir, de las bacterias que se encuentran en el intestino de forma natural sin ser patógenas), si bien este evento requeriría de la existencia de una secuencia muy parecida en el propio ADN de las bacterias expuestas al ADN foráneo.
La FDA estadounidense, autoridad competente en salud pública y alimentación, declaró que existe una posibilidad potencial de que esta transferencia tenga lugar a las células del epitelio gastrointestinal. Por tanto, ahora se exige la eliminación de marcadores de selección a antibióticos de las plantas transgénicas antes de su comercialización, lo que incrementa el coste de desarrollo pero elimina el riesgo de integración de ADN problemático.




%%%%%%%%%%%%%%%%%%%%%%%%%%%%%%%%%%%%%%
\subsection{Ingestión de "<ADN foráneo">}


Planta de tabaco transgénica expresando la luciferasa de la luciérnaga Photinus pyralis, enzima que permite la emisión de fluorescencia. 
Un aspecto que origina polémica es el empleo de ADN de una especie distinta a la del organismo transgénico; por ejemplo, que en maíz se incorpore un gen propio de una bacteria del suelo, y que este maíz esté destinado al consumo humano. No obstante, la incorporación de ADN de organismos bacterianos e incluso de virus sucede de forma constante en cualquier proceso de alimentación. De hecho, los procesos de preparación de alimento suelen fragmentar las moléculas de ADN de tal forma que el producto ingerido carece ya de secuencias codificantes (es decir, con genes completos capaces de codificar información.
Más aún, debido a que el ADN ingerido es desde un punto de vista químico igual ya provenga de una especie u otra, la especie del que proviene no tiene ninguna influencia. 


%----------------------------------------------------------
%
%   Target : tabaco
%
%----------------------------------------------------------

\begin{figure}[h]
\centering    % para centrar el parbox
\parbox{8cm}{
\centering    % para centrar el grafico dentro del parbox
\hypertarget{target.tabaco}{\includegraphics[width=6cm]{images/tabaco}}
\caption[Tabaco transgénico]{Planta de tabaco transgénica expresando la luciferasa de la luciérnaga \textit{Photinus pyralis}, enzima que permite la emisión de fluorescencia.}
\label{tabaco}
}
\end{figure}



La transformación de plántulas de cultivo in vitro suele realizarse con un cultivo de Agrobacterium tumefaciens en placas Petri con un medio de cultivo suplementado con antibióticos.
Esta preocupación se ha extendido en cuanto a los marcadores de resistencia a antibióticos que se cita en la sección anterior pero también respecto a la secuencia promotora de la transcripción que se sitúa en buena parte de las construcciones de ADN que se introducen en las plantas de interés alimentario, denominado promotor 35S y que procede del cauliflower mosaic virus (virus del mosaico de la coliflor). Puesto que este promotor produce expresión constitutiva (es decir, continua y en toda la planta) en varias especies, se sugirió su posible transferencia horizontal entre especies, así como su recombinación en plantas e incluso en virus, postulándose un posible papel en la generación de nuevas cepas virales.
No obstante, el propio genoma humano contiene en su secuencia multitud de repeticiones de ADN que proceden de retrovirus (un tipo de virus) y que, por definición, es ADN foráneo sin que haya resultado fatal en la evolución de la especie; estas repeticiones se calculan en unas 98.000
o, según otras fuentes, en 400.000.
Dado que, además, estas secuencias no tienen por qué ser adaptativas, es común que posean una tasa de mutación alta y que, en el transcurso de las generaciones, pierdan su función. Finalmente, puesto que el virus del mosaico de la coliflor está presente en el 10\% de nabos y coliflores no transgénicos, el ser humano ha consumido su promotor desde hace años sin efectos deletéreos.




%%%%%%%%%%%%%%%%%%%%%%%%%%%%%%%%%%%%%%
\subsection{Alergenicidad y toxicidad}


Se ha discutido el posible efecto como alérgenos de los derivados de alimentos transformados genéticamente; incluso, se ha sugerido su toxicidad. El concepto subyacente en ambos casos difiere: en el primero, una sustancia inocua podría dar lugar a la aparición de reacciones alérgicas en algunos individuos susceptibles, mientras que en el segundo su efecto deletéreo sería generalizado. Un estudio de gran repercusión al respecto fue publicado por Exwen y Pustzai en 1999. En él se indicaba que el intestino de ratas alimentadas con patatas genéticamente modificadas (expreando una aglutinina de Galanthus nivalis, que es una lectina) resultaba dañado severamente. 
No obstante, este estudio fue severamente criticado por varios investigadores por fallos en el diseño experimental y en el manejo de los datos. Por ejemplo, se incluyeron pocos animales en cada grupo experimental (lo que da lugar a una gran incertidumbre estadística), ni se analizó la composición química con precisión de las distintas variedades de patata empleadas, ni se incluyeron controles en los experimentos y finalmente, el análisis estadístico de los resultados era incorrecto.




\begin{wrapfigure}{O}{6cm}
\centering
\includegraphics[width=5cm]{images/plantulas}
\caption[Cultivo in vitro]{La transformación de plántulas de cultivo in vitro suele realizarse con un cultivo de Agrobacterium tumefaciens en placas Petri con un medio de cultivo suplementado con antibióticos.}
\label{plantulas}
\end{wrapfigure}




Estas críticas fueron rápidas: la comunidad científica respondió el mismo año recalcando las falencias del artículo; además, también se censuró a los autores la búsqueda de celebridad y la publicidad en medios periodísticos.
En cuanto a la evaluación toxicológica de los alimentos transgénicos, los resultados obtenidos por los científicos son contradictorios. Uno de los objetivos de estos trabajos es comprobar la pauta de función hepática, pues en este órgano se produce la detoxificación de sustancias en el organismo. Un estudio en ratón alimentado con soja resistente a glifosato encontró diferencias en la actividad celular de los hepatocitos, sugiriendo una modificación de la actividad metabólica al consumir transgénicos.
Estos estudios basados en ratones y soja fueron ratificados en cuanto a actividad pancreática y testículo. 
No obstante, otros científicos critican estos hallazgos debido a que no tuvieron en cuenta el método de cultivo, recolección y composición nutricional de la soja empleada; por ejemplo, la lína empleada era genéticamente bastante estable y fue cultivada en las mismas condiciones en el estudio de hepatocitos y páncreas, por lo que un elemento externo distinto al gen de resistencia a glifosato podría haber provocado su comportamiento al ser ingerido. Más aún, el contenido en isoflavonas de la variedad transgénica puede explicar parte de las modificaciones descritas en el intestino de la rata, y este elemento no se tuvo en cuenta puesto que ni se midió en el control ni en la variedad transgénica. 
Otros estudios independientes directamente no encontraron efecto alguno en el desarrollo testicular de ratones alimentados con soja resistente a glifosato o maíz Bt. 




%%%%%%%%%%%%%%%%%%%%%%%%%%%%%%%%%%%%%%%%%%%%%%%%%%%%%%%%%%%%%%%%%%%%%%%%%%
\section{Propiedad intelectual}


Un argumento frecuentemente esgrimido en contra de los alimentos trans\-génicos es el relacionado con la gestión de los derechos de propiedad intelectual y/o patentes, que obligan al pago de regalías por parte del agricultor al mejorador. Además, se alude al uso de estrategias moleculares que impiden la reutilización del tomate, es decir, el empleo de parte de la cosecha para cultivar en años sucesivos. Un ejemplo conocido de este último aspecto es la tecnología Terminator, englobado en las técnicas de restricción de uso (GURT), desarrollada por el Departamento de Agricultura de EE.UU. y la Delta and Pine Company en la década de 1990 y que aún no ha sido incorporada a cultivares comerciales, y por supuesto no está autorizada su venta. La restricción patentada opera mediante la inhibición de la germinación de las semillas, por ejemplo. 
Cabe destacar que el uso del vigor híbrido, una de las estrategias más frecuentes en mejora vegetal, en las variedades no tradicionales pero no transgénicas también imposibilita la reutilización de semillas. Este procedimiento se basa en el cruce de dos líneas puras que actúan como parentales, dando lugar a una progenie con un genotipo mixto que posee ventajas en cuanto a calidad y rendimiento. Debido a que la progenie es heterocigota para algunos genes, si se cruza consigo misma da lugar a una segunda generación muy variable por simple mendelismo, lo que resulta inadecuado para la producción agrícola. 


En cuanto a la posibilidad de patentar las plantas transgénicas, éstas pueden no someterse a una patente propiamente dicha, sino a unos derechos del obtentor, gestionados por la Unión Internacional para la Protección de Nuevas Variedades de Plantas. Brasil, España, Bolivia o Chile se encuentran en esa unión, siendo un total de 66 en diciembre de 2008 (entre los países no participantes destaca EE. UU.). 
Para la UPOV en su revisión de 1991, la ingeniería genética es una herramienta de introducción de variación genética en las variedades vegetales.
Bajo esta perspectiva, las plantas transgénicas son protegidas de forma equivalente a la de las variedades generadas por procedimientos convencionales; este hecho necesariamente exige la posibilidad de emplear variedades protegidas para agricultura de subsistencia e investigación científica. La UPOV también se pronunció en 2003 sobre las tecnologías de restricción de uso como la Terminator mencionada anteriormente: de acuerdo a la existencia de un marco legal de protección de las nuevas variedades, se indica que la aplicación de estas tecnologías no es necesaria. 






%%%%%%%%%%%%%%%%%%%%%%%%%%%%%%%%%%%%%%%%%%%%%%%%%%%%%%%%%%%%%%%%%%%%%%%%%%
\section{Impacto en los medios}


En 2009 fue prohibido el maíz transgénico MON 810 de Monsanto en Alemania por el Ministro Federal Aigner. Se anunció el alto inmediato del cultivo y comercialización del MON810, invocando, al igual que Francia, la cláusula de salvaguarda.
La Comisión Central Alemana para la Seguridad Biológica (ZKBS) consideró que la prohibición no estaba fundamentada científicamente. Más de 1600 científicos han apelado al Ministro Aigner que no se sacrifique una tecnología futura con gran potencial por razones de intereses políticos faltos de perspectiva. En un continente distinto, los políticos indios tratan de evitar el cultivo del brinjal Bt, a pesar de la evaluación positiva de seguridad por el Comité Indio para la Aprobación de la Ingeniería Genética.  



%----------------------------------------------------------
%
%   Enlace a  http://www.armoniafractal.com: 
%               
%   pinchando en la imagen mundoverde queremos enlazar con 
%   http://www.armoniafractal.com
%
%----------------------------------------------------------


\vskip2cm
\begin{center}
\href{http://www.sciencemag.org/}{\includegraphics[width=3cm]{images/mundoverde}}
\end{center}


%%%%%%%%%%%%%%%%%%%%%%%%%%%%%%%%%%%%%%%%%%%%%%%%%%%%%%%%%%%%%%%%%%%%%%%%%%

\newpage
\listoftables


\listoffigures




\addcontentsline{toc}{section}{Referencias}

\begin{thebibliography}{10}
\bibitem{amann}
H.~Amann.
\newblock {\em Ordinary differential equations. An introduction to nonlinear
  analysis}, volume~13 of {\em de Gruyter Studies in Mathematics}.
\newblock Walter de Gruyter \& Co., Berlin, 1990.

\bibitem{amrouche}
C.~Amrouche and V.~Girault.
\newblock On the existence and regularity of the solutions of {S}tokes problem
  in arbitrary dimension.
\newblock {\em Proc. Japan Acad. Ser. A Math. Sci.}, 67(5):171--175, 1991.

\bibitem{Beirao1}
H.~Beir\~ao~da Veiga.
\newblock Diffusion on viscous fluids. existence and asymptotic properties of
  solutions.
\newblock {\em Ann. Scuola Norm. Sup. Pisa Cl. Sci.}, 10(2):341--355, 1983.

\bibitem{braz2}
P.~Braz~e Silva, E.~Fern\'andez-Cara, and M.A. Rojas-Medar.
\newblock Vanishing viscosity for nonhomogeneous asymmetric fluid in
  $\mathbb{R}^3$.
\newblock {\em Journal of Mathematical Analysis and Applications},
  332(2):833--845, 2007.
\bibitem{braz1}
P.~Braz~e Silva and M.~A. Rojas-Medar.
\newblock Error estimates for semi-{G}alerkin approximations of nonhomogeneous
  incompressible fluids.
\newblock To appear, Journal of Mathematical Fluid Mechanics, 2007.
\bibitem{brezis}
H.~Brezis.
\newblock {\em Analyse fonctionnelle. Th\'eorie et applications}.
\newblock Collection Math\'ematiques Appliqu\'ees pour la Ma\^{i}trise. Masson,
  Paris, 1983.
\bibitem{evans}
L.~C. Evans.
\newblock {\em Partial Differential Equations}, volume~19 of {\em Graduate
  Studies in Mathematics}.
\newblock American Mathematical Society, Providence, RI, 1998.
\bibitem{enrique}
E.~Fern\'andez-Cara and F.~Guill\'en-Gonz\'alez.
\newblock The existence of nonhomogeneous, viscous and incompressible flow in
  unbounded domains.
\newblock {\em Comm. Partial Differential Equations}, 17(7-8):1253--1265, 1992.
\bibitem{frank}
D.~A. Frank and V.~I. Kamenetskii.
\newblock {\em Diffusion and Heat Transfer in Chemical Kinetics}.
\newblock Plenum Press, 1969.
\bibitem{kisko}
F.~Guill\'en-Gonz\'alez.
\newblock Sobre un modelo asint\'otico de difusi\'on de masa para fluidos
  incompresibles, viscoso y no homog\'eneos.
\newblock In {\em Prooceedings of the Third Catalan Days On Applied
  Mathematics}, pages 103--114, 1996.
\bibitem{kim}
J.~U. Kim.
\newblock Weak solutions of an initial-boundary value problems for an
  incompressible viscous fluid with nonnegative density.
\newblock {\em SIAM J. Math. Anal.}, 18(1):89--96, 1987.
\bibitem{olga1}
O.~A. Ladyzhenskaya.
\newblock {\em The mathematical theory of viscous incompressible flow},
  volume~2 of {\em Mathematics and its Applications}.
\newblock Gordon and Breach, New York-London-Paris, second edition, 1969.
\bibitem{pllions}
P.-L. Lions.
\newblock {\em Mathematical topics in fluid mechanics. Vol I: Incompressible
  models}, volume~3 of {\em Oxford Lecture Series in Mathematics and its
  Applications}.
\newblock The Clarendon Press, Oxford University Press, New York, 1996.
\bibitem{lukas2}
G.~Lukaszewicz.
\newblock {\em Micropolar fluids. Theory and applications}.
\newblock Modelling and Simulation in Science, Engineering \& Technology.
  Birkh{\"{a}}user Boston, Inc., Boston, MA, 1999.
\bibitem{okamoto}
H.~Okamoto.
\newblock On the equation of nonstationary stratified fluid motion: uniqueness
  and existence of the solutions.
\newblock {\em J. Fac. Sci. Univ. Tokyo Sect. IA Math.}, 30(3):615--643, 1984.
\bibitem{padula1}
M.~Padula.
\newblock An existence theorem for nonhomogeneous incompressible fluids.
\newblock {\em Rend. Circ. Mat. Palermo (2)}, 31(1):119--124, 1982.
\bibitem{panton}
R.~L. Panton.
\newblock {\em Incompressible flow}.
\newblock A Wiley-Interscience Publication. John Wiley \& Sons Inc., New York,
  1984.
  
\bibitem{reedsimon}
M.~Reed and B.~Simon.
\newblock {\em Methods of modern mathematical physics I. Functional analysis}.
\newblock Academic Press, Inc. [Harcout Brace Jovanovich, Publishers], New
  York, second edition, 1980.
  
\bibitem{salvi1}
R.~Salvi.
\newblock On the existence of weak solutions of a nonlinear mixed problem for
  nonhomogeneous fluids in a time dependent domain.
\newblock {\em Comment. Math. Univ. Carolin.}, 26(1):185--199, 1985.

\bibitem{salvi2}
R.~Salvi.
\newblock Error estimates for the spectral {G}alerkin approximations of the
  solutions of {N}avier-{S}tokes type equations.
\newblock {\em Glasgow Math. J.}, 31(2):199--211, 1989.
\bibitem{secchi2}
P.~Secchi.
\newblock On the initial value problem for the equations of motion of viscous
  incompressible fluids in the presence of diffusion.
\newblock {\em Boll. Un. Mat. Ital. B (6)}, 1(3):1117--1130, 1982.
\bibitem{simon4}
J.~Simon.
\newblock Compacts sets in the space ${L}^p (0,t;{B})$.
\newblock {\em Ann. Mat. Pura Appl. (4)}, 146:65--96, 1987.
\bibitem{simon5}
J.~Simon.
\newblock Existencia de soluci\'on del problema de {N}avier-{S}tokes con
  densidad variable. (spanish) [existence of solution for the variable density
  {N}avier-{S}tokes problem].
\newblock Lectures at the University of Sevilla, Spain, 1989.
\bibitem{simon3}
J.~Simon.
\newblock Nonhomogeneous viscous incompressible fluids: existence of velocity,
  density, and pressure.
\newblock {\em SIAM J. Math. Anal.}, 21(5):1093--1117, 1990.
\bibitem{chia}
C.-S. Yih.
\newblock {\em Dynamics of nonhomogeneous fluids}.
\newblock The Macmillan Co., 1965.
\end{thebibliography}

\end{document}
%%%%%%%%%%%%%%%%%%%%%%%%%%%%%%%%%%%%%%%%%%%%%%%%%%%%%%%%%